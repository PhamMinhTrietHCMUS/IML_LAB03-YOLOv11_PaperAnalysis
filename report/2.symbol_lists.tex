\chapter*{\MakeUppercase{Danh sách các kí tự viết tắt, thuật ngữ}}
\addcontentsline{toc}{chapter}{{\bf \MakeUppercase{Danh sách các kí tự viết tắt, thuật ngữ}}}

\begin{longtable}{p{0.25\textwidth} p{0.65\textwidth}}
\toprule
\textbf{Thuật ngữ} & \textbf{Mô tả} \\
\midrule
\endfirsthead
\caption[]{Danh sách các kí tự viết tắt (tiếp theo)} \\
\toprule
\textbf{Thuật ngữ} & \textbf{Mô tả} \\
\midrule
\endhead
\bottomrule
\endfoot
AI & Trí tuệ nhân tạo (Artificial Intelligence) \\
CV & Thị giác máy tính (Computer Vision) \\
CNN & Mạng nơ-ron tích chập (Convolutional Neural Network) \\
YOLO & Bạn chỉ nhìn một lần (You Only Look Once) \\
SOTA & Tiên tiến nhất (State-of-the-Art) \\
FPS & Số khung hình trên giây (Frames Per Second) \\
Backbone & Mạng xương sống (Trích xuất đặc trưng) \\
Neck & Cổ mạng (Hợp nhất đặc trưng) \\
Head & Đầu dò (Dự đoán kết quả) \\
SPPF & Spatial Pyramid Pooling - Fast \\
C3K2 & Khối CSP với kích thước kernel tùy chỉnh (Module mới trong YOLOv11) \\
C2PSA & Cross-Level Pyramid Slice Attention (Cơ chế chú ý trong YOLOv11) \\
CSP & Cross Stage Partial (Kiến trúc mạng) \\
PANet & Path Aggregation Network \\
IoU & Giao trên hợp (Intersection over Union) \\
mAP & Độ chính xác trung bình (mean Average Precision) \\
NMS & Triệt tiêu phi cực đại (Non-Maximum Suppression) \\
SiLU & Sigmoid Linear Unit (Hàm kích hoạt) \\
DFL & Distribution Focal Loss \\
CIoU & Complete Intersection over Union \\
BCE & Binary Cross Entropy
\end{longtable}

\newpage

\chapter*{\MakeUppercase{Danh sách các kí tự}}
\addcontentsline{toc}{chapter}{\MakeUppercase{Danh sách các kí tự}}

\begin{longtable}{p{0.2\textwidth} p{0.7\textwidth}}
\toprule
\textbf{Kí hiệu} & \textbf{Mô tả} \\
\midrule
\endfirsthead
\caption[]{Danh sách các kí hiệu toán học (tiếp theo)} \\
\toprule
\textbf{Kí hiệu} & \textbf{Mô tả} \\
\midrule
\endhead
\bottomrule
\endfoot
$\mathbb{R}$ & Tập số thực \\
$\mathcal{I}$ & Hình ảnh đầu vào \\
$H, W, C$ & Chiều cao, Chiều rộng và Số kênh của ảnh/tensor \\
$\mathbf{X}$ & Tensor đặc trưng đầu vào \\
$\mathbf{Y}$ & Tensor đặc trưng đầu ra \\
$\mathbf{K}$ & Bộ lọc tích chập (Kernel) \\
$Q, K, V$ & Các ma trận Truy vấn (Query), Khóa (Key) và Giá trị (Value) trong cơ chế chú ý \\
$\mathcal{F}$ & Hàm ánh xạ của mạng nơ-ron \\
$\mathcal{O}$ & Tensor dự đoán đầu ra \\
$\mathcal{L}_{total}$ & Hàm mất mát tổng thể \\
$\mathcal{L}_{box}$ & Hàm mất mát hồi quy hộp giới hạn \\
$\mathcal{L}_{cls}$ & Hàm mất mát phân loại \\
$\mathcal{L}_{dfl}$ & Hàm mất mát phân phối tiêu điểm \\
$B_{p}$ & Hộp giới hạn dự đoán (Predicted Bounding Box) \\
$B_{gt}$ & Hộp giới hạn nhãn (Ground Truth Bounding Box) \\
$C$ & Điểm tin cậy (Confidence score) hoặc Nhãn lớp \\
$\sigma(.)$ & Hàm kích hoạt Sigmoid \\
$\rho$ & Khoảng cách Euclidean \\
$\theta$ & Tập hợp tham số trọng số của mô hình \\
\end{longtable}