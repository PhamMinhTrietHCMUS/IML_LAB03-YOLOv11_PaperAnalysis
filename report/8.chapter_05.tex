\chapter{Thực nghiệm và Đánh giá}

Chương này trình bày kết quả thực nghiệm đã triển khai, mô tả bộ dữ liệu và các tiêu chí đánh giá được sử dụng để kiểm chứng hiệu quả của kiến trúc YOLOv11. Mục tiêu của phần thực nghiệm là so sánh hiệu suất của YOLOv11 với các phiên bản tiền nhiệm (YOLOv8, YOLOv9) trên bài toán nhận diện cử chỉ tay.

\section{Bộ dữ liệu (Dataset)}
Để đánh giá khả năng phát hiện đối tượng trong bối cảnh thực tế với các đặc trưng hình học gần giống nhau, nghiên cứu sử dụng bộ dữ liệu \textbf{Rock Paper Scissors} được lưu trữ trên Roboflow.

\begin{itemize}
    \item \textbf{Nguồn dữ liệu:} Roboflow Universe \footnote{\url{https://universe.roboflow.com/roboflow-58fyf/rock-paper-scissors-sxsw}}.
    \item \textbf{Số lượng lớp (Classes):} 3 lớp (Rock, Paper, Scissors).
    \item \textbf{Đặc điểm:} Bộ dữ liệu bao gồm các hình ảnh bàn tay mô phỏng ba cử chỉ của trò chơi "Oẳn tù tì". Đây là bài toán thử thách khả năng của mô hình trong việc phân biệt các chi tiết hình dạng ngón tay và tư thế tay, vốn có sự tương đồng cao về màu sắc da và bối cảnh.
    \item \textbf{Phân chia dữ liệu:} Dữ liệu được chia theo tỷ lệ tiêu chuẩn:
    \begin{itemize}
        \item \textbf{Train set:} 70\% (2196 ảnh - Dùng để huấn luyện trọng số).
        \item \textbf{Validation set:} 20\% (604 ảnh - Dùng để tinh chỉnh siêu tham số và đánh giá trong quá trình huấn luyện).
        \item \textbf{Test set:} 10\% (329 ảnh - Dùng để đánh giá hiệu suất cuối cùng).
    \end{itemize}
\end{itemize}

\section{Môi trường và Cấu hình Thực nghiệm}

\subsection{Phần cứng}
Quá trình huấn luyện và kiểm thử dự kiến sẽ được thực hiện trên môi trường máy tính cá nhân hiệu năng cao (High-End PC) với cấu hình hỗ trợ GPU thế hệ mới để tối ưu hóa thời gian huấn luyện:
\begin{itemize}
    \item \textbf{GPU:} NVIDIA RTX 5070 WINDFORCE OC SFF 12G (12 GB VRAM).
    \item \textbf{CPU:} Intel Core i5-14600K.
    \item \textbf{RAM:} Không giới hạn (Đảm bảo dung lượng hệ thống không là điểm nghẽn).
\end{itemize}

\subsection{Phần mềm và Thư viện}
\begin{itemize}
    \item \textbf{Ngôn ngữ:} Python 3.13+.
    \item \textbf{Framework:} PyTorch (phiên bản hỗ trợ CUDA mới nhất).
    \item \textbf{Thư viện YOLO:} Ultralytics (hỗ trợ YOLOv8, YOLOv11) và repository chính thức của YOLOv9.
    \item \textbf{Cấu hình huấn luyện:} Các tham số siêu hình (hyperparameters) được thiết lập cố định cho các mô hình để đảm bảo công bằng:
    \begin{itemize}
        \item \textbf{Epochs:} 100.
        \item \textbf{Batch size:} 16 hoặc 32 (tự động điều chỉnh dựa trên VRAM).
        \item \textbf{Image Size:} $640 \times 640$.
        \item \textbf{Optimizer:} Tự động (Auto - thường là SGD hoặc AdamW tùy ngữ cảnh).
    \end{itemize}
\end{itemize}

Câu lệnh huấn luyện mẫu cho YOLOv11 được thực hiện như sau:
\begin{verbatim}
yolo detect train data=data.yaml model=yolo11s.pt epochs=100 imgsz=640
\end{verbatim}
\subsection{Thời gian Huấn luyện}

Bảng \ref{tab:training_time} so sánh tổng thời gian huấn luyện và thời gian trung bình mỗi epoch của ba mô hình YOLO:

\begin{table}[H]
\centering
\caption{So sánh thời gian huấn luyện (100 epochs)}
\label{tab:training_time}
\begin{tabular}{lccc}
\hline
\textbf{Mô hình} & \textbf{Tổng thời gian (giây)} & \textbf{Tổng thời gian (giờ)} & \textbf{TB/epoch (giây)} \\
\hline
YOLOv8s  & 2066.24 & 0.574 & 20.66 \\
YOLOv9s  & 3085.47 & 0.857 & 30.85 \\
YOLOv11s & 2355.41 & 0.654 & 23.55 \\
\hline
\end{tabular}
\end{table}

\section{So sánh và Đánh giá}
Nghiên cứu sẽ thực hiện so sánh đối chứng giữa ba thế hệ YOLO gần nhất để làm rõ sự cải tiến của kiến trúc YOLOv11.

\subsection{Các mô hình đối chứng}
\begin{enumerate}
    \item \textbf{YOLOv8 (Baseline):} Phiên bản ổn định và phổ biến nhất hiện nay, sử dụng khối C2f.
    \item \textbf{YOLOv9:} Phiên bản tập trung vào kiến trúc GELAN và PGI.
    \item \textbf{YOLOv11 (Đề xuất):} Phiên bản mới nhất sử dụng khối C3K2 và cơ chế chú ý C2PSA.
\end{enumerate}
Để đảm bảo tính công bằng và phù hợp với câu lệnh huấn luyện đã thiết lập, các mô hình sẽ được so sánh ở cùng một kích thước \textbf{Small (s)} (ví dụ: \texttt{yolov8s.pt}, \texttt{yolov9s.pt}, \texttt{yolo11s.pt}). Phân khúc Small cung cấp sự cân bằng tốt giữa độ chính xác và tốc độ, phù hợp với phần cứng thí nghiệm.

\subsection{Các chỉ số đánh giá (Evaluation Metrics)}
Hiệu suất của các mô hình sẽ được đo lường qua các chỉ số định lượng sau:

\begin{itemize}
    \item \textbf{mAP@50 (Mean Average Precision at IoU=0.5):} Độ chính xác trung bình khi ngưỡng IoU là 0.5. Chỉ số này đánh giá khả năng phát hiện đúng đối tượng.
    \item \textbf{mAP@50-95:} Chỉ số khắt khe hơn, là trung bình của mAP tại các ngưỡng IoU từ 0.5 đến 0.95 (bước nhảy 0.05). Chỉ số này đánh giá độ chính xác của vị trí hộp giới hạn (localization accuracy).
    \item \textbf{Params (Parameters):} Tổng số lượng tham số của mô hình (đơn vị: triệu).
    \item \textbf{FLOPs (Floating Point Operations):} Số phép tính dấu phẩy động (đơn vị: Giga). Đánh giá chi phí tính toán.
    \item \textbf{FPS (Frames Per Second):} Tốc độ xử lý thực tế trên GPU RTX 5070.
\end{itemize}

\section{Kết quả Thực nghiệm}

Sau khi hoàn thành quá trình huấn luyện 100 epochs cho cả ba mô hình trên bộ dữ liệu Rock-Paper-Scissors, các kết quả đánh giá trên tập test (329 ảnh) được tổng hợp như sau:

\subsection{Kết quả So sánh Tổng quan}

Bảng \ref{tab:comparison} tổng hợp các chỉ số đánh giá chính của ba mô hình:

\begin{table}[H]
\centering
\caption{So sánh hiệu suất các mô hình YOLO trên tập Test}
\label{tab:comparison}
\begin{tabular}{lcccccc}
\hline
\textbf{Model} & \textbf{Precision} & \textbf{Recall} & \textbf{mAP@50} & \textbf{mAP@50-95} & \textbf{F1-Score} \\
\hline
YOLOv8s  & \textbf{0.9612} & 0.9069 & 0.9541 & \textbf{0.8084} & 0.9333 \\
YOLOv9s  & 0.9546 & 0.9144 & 0.9496 & 0.8078 & 0.9340 \\
YOLOv11s & 0.9481 & \textbf{0.9241} & \textbf{0.9548} & 0.8075 & \textbf{0.9360} \\
\hline
\end{tabular}
\end{table}


\subsection{Đánh giá Quá trình Huấn luyện}

Phân tích chi tiết các đường cong loss trong quá trình huấn luyện 100 epochs cho thấy những đặc điểm quan trọng về hành vi hội tụ của từng mô hình.

\subsubsection{Phân tích Loss Functions}

Bảng \ref{tab:final_loss} tổng hợp giá trị loss cuối cùng (epoch 100) của ba mô hình:

\begin{table}[H]
\centering
\caption{So sánh giá trị Loss tại epoch 100}
\label{tab:final_loss}
\begin{tabular}{lccc|ccc}
\hline
\textbf{Mô hình} & \multicolumn{3}{c|}{\textbf{Training Loss}} & \multicolumn{3}{c}{\textbf{Validation Loss}} \\
\cline{2-7}
 & \textbf{Box} & \textbf{Cls} & \textbf{DFL} & \textbf{Box} & \textbf{Cls} & \textbf{DFL} \\
\hline
YOLOv8s  & \textbf{0.407} & \textbf{0.231} & \textbf{0.858} & \textbf{0.847} & 0.468 & \textbf{1.126} \\
YOLOv9s  & 0.460 & 0.233 & 0.980 & 0.820 & \textbf{0.417} & 1.363 \\
YOLOv11s & 0.458 & 0.255 & 0.878 & 0.819 & 0.469 & 1.080 \\
\hline
\end{tabular}
\end{table}

\subsubsection{Biểu đồ Quá trình Huấn luyện}

Hình \ref{fig:training_yolov8}, \ref{fig:training_yolov9}, và \ref{fig:training_yolov11} trình bày chi tiết quá trình huấn luyện của từng mô hình:

\begin{figure}[H]
    \centering
    \includegraphics[width=\textwidth]{../runs/detect/yolov8/results.png}
    \caption{Quá trình huấn luyện YOLOv8s qua 100 epochs}
    \label{fig:training_yolov8}
\end{figure}

\begin{figure}[H]
    \centering
    \includegraphics[width=\textwidth]{../runs/detect/yolov9/results.png}
    \caption{Quá trình huấn luyện YOLOv9s qua 100 epochs}
    \label{fig:training_yolov9}
\end{figure}

\begin{figure}[H]
    \centering
    \includegraphics[width=\textwidth]{../runs/detect/yolov11/results.png}
    \caption{Quá trình huấn luyện YOLOv11s qua 100 epochs}
    \label{fig:training_yolov11}
\end{figure}

\subsubsection{Nhận xét Chi tiết}

\textbf{1. Box Loss (Localization Loss):}

Box Loss đo lường độ chính xác của việc định vị bounding box. Phân tích cho thấy:

\begin{itemize}
    \item \textbf{YOLOv8s:} Đạt train/box\_loss thấp nhất (0.407), giảm từ 1.28 xuống 0.41 (-68\%). Validation loss ổn định ở 0.847.
    \item \textbf{YOLOv9s:} Train/box\_loss = 0.460, giảm từ 1.25 xuống 0.46 (-63\%). Tuy nhiên, validation loss vẫn đang có xu hướng giảm (0.820).
    \item \textbf{YOLOv11s:} Train/box\_loss = 0.458, giảm từ 1.30 xuống 0.46 (-65\%). \textbf{Đáng chú ý, đường cong train/box\_loss của YOLOv11s vẫn có xu hướng giảm đều đặn tại epoch 100}, cho thấy mô hình chưa hoàn toàn hội tụ và có thể cải thiện thêm nếu tăng số epochs.
\end{itemize}

\textbf{2. Classification Loss:}

Classification Loss đánh giá khả năng phân loại đối tượng:

\begin{itemize}
    \item \textbf{YOLOv8s:} Đạt train/cls\_loss thấp nhất (0.231), hội tụ nhanh và ổn định từ epoch 80.
    \item \textbf{YOLOv9s:} Train/cls\_loss = 0.233, gần tương đương YOLOv8s. Val/cls\_loss thấp nhất (0.417).
    \item \textbf{YOLOv11s:} Train/cls\_loss cao hơn một chút (0.255), nhưng vẫn trong khoảng chấp nhận được.
\end{itemize}

\textbf{3. Distribution Focal Loss (DFL):}

DFL Loss liên quan đến việc tinh chỉnh vị trí bounding box:

\begin{itemize}
    \item \textbf{YOLOv8s:} Đạt train/dfl\_loss thấp nhất (0.858) và val/dfl\_loss tốt nhất (1.126).
    \item \textbf{YOLOv9s:} DFL loss cao nhất ở cả train (0.980) và validation (1.363), phản ánh kiến trúc GELAN xử lý localization khác biệt.
    \item \textbf{YOLOv11s:} DFL loss ở mức trung bình (0.878 train, 1.080 val).
\end{itemize}

\subsubsection{Đánh giá Khả năng Hội tụ}

\textbf{Phân tích xu hướng hội tụ:}

\begin{enumerate}
    \item \textbf{YOLOv8s - Hội tụ tốt:} Các đường loss đều phẳng (plateau) từ epoch 85-100, cho thấy mô hình đã hội tụ gần như hoàn toàn. Training thêm epochs sẽ không cải thiện đáng kể.
    
    \item \textbf{YOLOv9s - Hội tụ chậm:} Do kiến trúc phức tạp hơn với GELAN và PGI, YOLOv9s cần nhiều epochs hơn để hội tụ. Validation loss vẫn có xu hướng giảm nhẹ, đặc biệt là val/cls\_loss.
    
    \item \textbf{YOLOv11s - Tiềm năng cải thiện:} 
    \begin{itemize}
        \item Train/box\_loss vẫn đang giảm với gradient đáng kể tại epoch 100 (từ 0.48 ở epoch 95 xuống 0.46 ở epoch 100).
        \item Metrics (mAP50, mAP50-95) vẫn có dao động nhẹ, chưa hoàn toàn ổn định.
        \item \textbf{Khuyến nghị:} Có thể tăng lên 120-150 epochs để đạt hiệu suất tối ưu.
    \end{itemize}
\end{enumerate}

\textbf{Hiện tượng quan sát được:}

\begin{itemize}
    \item \textbf{Learning Rate Decay:} Cả ba mô hình đều sử dụng cosine annealing scheduler, với learning rate giảm từ ~0.0014 xuống ~0.00003 tại epoch 100. Điều này giải thích sự ổn định của các metrics ở giai đoạn cuối.
    
    \item \textbf{Không có Overfitting:} Gap giữa train loss và validation loss duy trì ổn định, không có dấu hiệu validation loss tăng trong khi train loss giảm. Điều này chứng tỏ bộ dữ liệu và augmentation phù hợp.
    
    \item \textbf{Early Stopping không kích hoạt:} Với patience mặc định, không có mô hình nào dừng sớm, cho thấy tất cả đều tiếp tục cải thiện (dù chậm) trong suốt 100 epochs.
\end{itemize}

\begin{table}[H]
\centering
\caption{Đánh giá mức độ hội tụ của các mô hình}
\label{tab:convergence}
\begin{tabular}{lccc}
\hline
\textbf{Tiêu chí} & \textbf{YOLOv8s} & \textbf{YOLOv9s} & \textbf{YOLOv11s} \\
\hline
Mức độ hội tụ & Hoàn toàn & Gần hoàn toàn & Chưa hoàn toàn \\
Xu hướng Box Loss & Phẳng & Giảm nhẹ & Còn giảm \\
Tiềm năng cải thiện & Thấp & Trung bình & Cao \\
Epochs khuyến nghị & 100 (đủ) & 110-120 & 120-150 \\
\hline
\end{tabular}
\end{table}

\textbf{Kết luận:} Mặc dù YOLOv11s đạt F1-Score và mAP@50 cao nhất trong thực nghiệm này, việc train dừng ở epoch 100 có thể chưa khai thác hết tiềm năng của mô hình. Đường cong box\_loss vẫn đang giảm cho thấy cơ chế chú ý C2PSA cần nhiều thời gian hơn để học các đặc trưng phức tạp. Trong các ứng dụng thực tế, việc tăng số epochs cho YOLOv11 có thể mang lại cải thiện đáng kể về độ chính xác định vị (mAP@50-95).

\subsection{Phân tích Kết quả}

\begin{enumerate}
    \item \textbf{Về độ chính xác (Precision):}
    YOLOv8s đạt Precision cao nhất (96.12\%), cao hơn YOLOv9s (95.46\%) và YOLOv11s (94.81\%). Điều này cho thấy YOLOv8s có khả năng giảm thiểu False Positives tốt hơn trong bộ dữ liệu này.

    \item \textbf{Về độ phủ (Recall):}
    YOLOv11s dẫn đầu với Recall 92.41\%, vượt trội so với YOLOv9s (91.44\%) và YOLOv8s (90.69\%). Cơ chế chú ý C2PSA đã giúp mô hình phát hiện được nhiều đối tượng thực tế hơn (giảm False Negatives).

    \item \textbf{Về mAP@50:}
    YOLOv11s đạt mAP@50 cao nhất (95.48\%), nhỉnh hơn YOLOv8s (95.41\%) và YOLOv9s (94.96\%). Đây là chỉ số quan trọng đánh giá khả năng phát hiện đối tượng tổng thể.

    \item \textbf{Về mAP@50-95:}
    Cả ba mô hình đều đạt kết quả rất gần nhau (~80.8\%), cho thấy độ chính xác định vị (localization) tương đương. YOLOv8s nhỉnh hơn một chút với 80.84\%.

    \item \textbf{Về F1-Score (Cân bằng tổng thể):}
    YOLOv11s đạt F1-Score cao nhất (93.60\%), thể hiện sự cân bằng tốt nhất giữa Precision và Recall. Đây là chỉ số quan trọng nhất để đánh giá hiệu suất tổng thể.
\end{enumerate}


\subsection{Kết quả Chi tiết theo Lớp}

Bảng \ref{tab:per_class} trình bày kết quả đánh giá chi tiết cho từng lớp đối tượng:

\begin{table}[H]
\centering
\caption{Kết quả đánh giá theo từng lớp (YOLOv11s)}
\label{tab:per_class}
\begin{tabular}{lcccc}
\hline
\textbf{Class} & \textbf{Images} & \textbf{Instances} & \textbf{Precision} & \textbf{Recall} \\
\hline
Paper    & 73 & 73 & 0.941 & 0.881 \\
Rock     & 66 & 75 & 0.933 & 0.920 \\
Scissors & 67 & 69 & 0.970 & 0.971 \\
\hline
\textbf{All} & \textbf{329} & \textbf{217} & \textbf{0.948} & \textbf{0.924} \\
\hline
\end{tabular}
\end{table}

\subsection{Nhận xét}

\begin{itemize}
    \item \textbf{Kết quả vượt mong đợi:} Cả ba mô hình đều đạt hiệu suất rất cao (>93\% F1-Score), chứng tỏ kiến trúc YOLO rất phù hợp với bài toán nhận diện cử chỉ tay.
    \item \textbf{YOLOv11 thể hiện sự cân bằng tốt:} Mặc dù không dẫn đầu ở mọi chỉ số, YOLOv11s có F1-Score cao nhất, thể hiện sự cân bằng tối ưu giữa Precision và Recall.
    \item \textbf{Cải tiến của C2PSA:} Recall cao hơn của YOLOv11s chứng tỏ cơ chế chú ý đã giúp mô hình "nhìn thấy" nhiều đối tượng hơn, đặc biệt trong các trường hợp khó.
    \item \textbf{Sự khác biệt nhỏ:} Khoảng cách hiệu suất giữa các mô hình là rất nhỏ (<1\%), cho thấy cả ba thế hệ YOLO đều đã đạt mức độ tinh chỉnh cao.
\end{itemize}

\subsection{Trực quan hóa Kết quả}

\subsubsection{Biểu đồ So sánh Tổng quan}

Hình \ref{fig:model_comparison} và \ref{fig:comparison_table} trình bày kết quả so sánh trực quan giữa ba mô hình. Lưu ý rằng trục Y được offset từ 60\% để hiển thị rõ hơn sự khác biệt giữa các giá trị trong khoảng 90-99\%.

\begin{figure}[H]
    \centering
    \includegraphics[width=\textwidth]{../evaluation_results/model_comparison.png}
    \caption{So sánh hiệu suất các mô hình YOLO trên test set (Y-axis từ 60\%)}
    \label{fig:model_comparison}
\end{figure}

\begin{figure}[H]
    \centering
    \includegraphics[width=\textwidth]{../evaluation_results/radar_charts_combined.png}
    \caption{Radar chart riêng cho  các model}
    \label{fig:model_radar_charts}
\end{figure}


\begin{table}[H]
\centering
\caption{Bảng so sánh chi tiết (giá trị in đậm = tốt nhất)}
\label{tab:detailed_comparison}
\begin{tabular}{lccc}
\hline
\textbf{Metric} & \textbf{YOLOv8} & \textbf{YOLOv9} & \textbf{YOLOv11} \\
\hline
Precision (P) & \textbf{0.9612} & 0.9546 & 0.9481 \\
Recall (R) & 0.9069 & 0.9144 & \textbf{0.9241} \\
mAP@50 & 0.9541 & 0.9496 & \textbf{0.9548} \\
mAP@50-95 & \textbf{0.8084} & 0.8078 & 0.8075 \\
F1-Score & 0.9333 & 0.9340 & \textbf{0.9360} \\
\hline
\end{tabular}
\end{table}



Kết quả thực nghiệm cho thấy cả ba mô hình đều đạt hiệu suất rất cao (>93\%) trên bộ dữ liệu Rock-Paper-Scissors, với sự khác biệt nhỏ giữa các phiên bản.

\textbf{YOLOv11s} thể hiện sự cân bằng tốt nhất với F1-Score cao nhất (0.9360) và mAP@50 tốt nhất (0.9548), đồng thời dẫn đầu về Recall (0.9241) - cho thấy khả năng phát hiện đối tượng toàn diện hơn nhờ cơ chế chú ý C2PSA. Tuy nhiên, mô hình này có Precision thấp nhất (0.9481), nghĩa là tỷ lệ dự đoán dương tính giả (False Positives) cao hơn một chút.

\textbf{YOLOv8s} nổi bật với Precision cao nhất (0.9612) và mAP@50-95 tốt nhất (0.8084), thể hiện độ chính xác định vị (localization) vượt trội. Đây là lựa chọn phù hợp khi ưu tiên giảm thiểu False Positives.

\textbf{YOLOv9s} đạt kết quả trung gian ở hầu hết các chỉ số, không nổi trội ở bất kỳ chỉ số nào nhưng cũng không kém nhất. Kiến trúc GELAN và PGI của YOLOv9 chưa thể hiện lợi thế rõ rệt trên bộ dữ liệu này.

\textbf{Kết luận:} YOLOv11s là lựa chọn tối ưu nhất cho bài toán này nhờ sự cân bằng giữa Precision và Recall, đặc biệt phù hợp với các ứng dụng yêu cầu phát hiện toàn diện. Tuy nhiên, nếu ưu tiên độ chính xác tuyệt đối và giảm False Positives, YOLOv8s vẫn là lựa chọn đáng cân nhắc.

\subsubsection{Confusion Matrix}
Ma trận nhầm lẫn của YOLOv11s (Hình \ref{fig:confusion_yolov11}) cho thấy mô hình phân loại rất chính xác giữa các cử chỉ:

\begin{figure}[H]
    \centering
    \includegraphics[width=0.75\textwidth]{../evaluation_results/YOLOv11/confusion_matrix_normalized.png}
    \caption{Ma trận nhầm lẫn chuẩn hóa của YOLOv11s}
    \label{fig:confusion_yolov11}
\end{figure}


Ma trận nhầm lẫn chuẩn hóa của YOLOv11s (Hình \ref{fig:confusion_yolov11}) tiết lộ những thông tin quan trọng về hành vi của mô hình:

\textbf{Hiệu suất phân loại theo lớp:}
\begin{itemize}
    \item \textbf{Scissors (Kéo):} Đạt độ chính xác cao nhất (97\%), cho thấy đặc trưng hai ngón tay duỗi thẳng rất dễ nhận diện.
    \item \textbf{Rock (Búa):} Độ chính xác thấp nhất (93\%), phản ánh độ khó trong việc phát hiện cử chỉ này.
    \item \textbf{Paper (Bao):} Đạt 90\% độ chính xác, ở mức trung bình.
\end{itemize}

\textbf{Vấn đề nhầm lẫn với Background (quan trọng):}

Một phát hiện đáng chú ý là cử chỉ \textbf{Rock có tỷ lệ nhầm lẫn với background cao nhất (56\%)}, gấp hơn đôi so với Paper (22\%) và Scissors (22\%). Điều này có thể giải thích bởi:

\begin{enumerate}
    \item \textbf{Đặc trưng ít nổi bật:} Rock là bàn tay nắm chặt, tạo thành khối tròn đặc với ít chi tiết hình học phân biệt (không có ngón tay duỗi ra như Paper hay Scissors).
    \item \textbf{Khó phân biệt với vùng nền:} Hình dạng tròn, đặc của nắm đấm dễ bị nhầm lẫn với các vật thể tròn hoặc vùng tối trong background.
    \item \textbf{Diện tích nhỏ hơn:} Khi nắm lại, bàn tay có diện tích bounding box nhỏ hơn, làm giảm số lượng feature được trích xuất.
\end{enumerate}

\textbf{Nhầm lẫn giữa các lớp:}

Các cử chỉ ít bị nhầm lẫn với nhau (< 5\%), chứng tỏ mô hình phân biệt tốt giữa các hình dạng bàn tay. Tuy nhiên, thách thức chính nằm ở việc \textit{phát hiện sự tồn tại} của đối tượng Rock trong ảnh (detection) hơn là \textit{phân loại sai} (classification).

\textbf{Hướng cải tiến:} Để cải thiện hiệu suất trên lớp Rock, có thể:
\begin{itemize}
    \item Tăng cường dữ liệu (augmentation) đặc biệt cho lớp Rock với các background đa dạng.
    \item Điều chỉnh class weights trong loss function để mô hình chú ý hơn đến lớp khó này.
    \item Thu thập thêm dữ liệu Rock với các góc chụp và điều kiện ánh sáng khác nhau.
\end{itemize}

\subsubsection{Đường cong Precision-Recall}

Hình \ref{fig:pr_curves} so sánh đường cong PR của ba mô hình:

\begin{figure}[H]
    \centering
    \begin{subfigure}[b]{0.48\textwidth}
        \centering
        \includegraphics[width=\textwidth]{../evaluation_results/YOLOv8/BoxPR_curve.png}
        \caption{YOLOv8s}
    \end{subfigure}
    \hfill
    \begin{subfigure}[b]{0.48\textwidth}
        \centering
        \includegraphics[width=\textwidth]{../evaluation_results/YOLOv9/BoxPR_curve.png}
        \caption{YOLOv9s}
    \end{subfigure}

    \vspace{0.5cm}

    \begin{subfigure}[b]{0.48\textwidth}
        \centering
        \includegraphics[width=\textwidth]{../evaluation_results/YOLOv11/BoxPR_curve.png}
        \caption{YOLOv11s}
    \end{subfigure}
    \caption{Đường cong Precision-Recall của các mô hình}
    \label{fig:pr_curves}
\end{figure}


Đường cong Precision-Recall (PR Curve) là công cụ quan trọng để đánh giá khả năng cân bằng giữa độ chính xác và độ phủ của mô hình phát hiện đối tượng. Diện tích dưới đường cong (AUC) chính là chỉ số mAP@50.

\textbf{So sánh hiệu suất tổng quan (mAP@50):}
\begin{itemize}
    \item \textbf{YOLOv11s:} 0.955 (cao nhất)
    \item \textbf{YOLOv8s:} 0.954
    \item \textbf{YOLOv9s:} 0.950 (thấp nhất)
\end{itemize}

\textbf{Đặc điểm chung của các đường cong:}

Cả ba mô hình đều thể hiện đường cong PR \textit{lý tưởng} với các đặc điểm:
\begin{enumerate}
    \item \textbf{Vùng plateau rộng:} Precision duy trì ở mức gần 1.0 (>0.95) khi Recall tăng từ 0 đến ~0.8-0.9. Điều này chứng tỏ mô hình có khả năng phát hiện nhiều đối tượng mà vẫn giữ độ chính xác cao.
    \item \textbf{"Đầu gối" xuất hiện muộn:} Precision chỉ bắt đầu giảm mạnh khi Recall > 0.8, cho thấy mô hình có ngưỡng confidence tốt, không tạo ra nhiều False Positives khi cố gắng tăng độ phủ.
    \item \textbf{Độ nhất quán cao:} Các đường cong của 3 lớp (Paper, Rock, Scissors) gần sát nhau, thể hiện hiệu suất cân bằng giữa các lớp.
\end{enumerate}

\textbf{Phân tích theo từng lớp:}

\begin{itemize}
    \item \textbf{Scissors (Kéo):} Dẫn đầu ở YOLOv8s (0.977) và YOLOv11s (0.983), đạt mAP cao nhất trong 3 lớp. Đường cong gần như hoàn hảo phản ánh đặc trưng hai ngón tay duỗi rất dễ nhận diện và ít bị nhầm lẫn.

    \item \textbf{Rock (Búa):} Có sự cải thiện đáng kể từ YOLOv8s (0.948) lên YOLOv9s (0.958) và YOLOv11s (0.956). Tuy nhiên, như đã phân tích trong ma trận nhầm lẫn, Rock vẫn gặp khó khăn với background confusion (56\%). Đường cong PR cao không phản ánh vấn đề này vì PR chỉ đo lường \textit{độ chính xác của các detection được đưa ra}, không đo lường \textit{số lượng đối tượng bị bỏ sót} (False Negatives với background).

    \item \textbf{Paper (Bao):} Kết quả ổn định nhất, dao động từ 0.926-0.942 giữa các mô hình. YOLOv9s đạt kết quả tốt nhất (0.942) cho lớp này.
\end{itemize}

\textbf{So sánh giữa các mô hình:}

\begin{itemize}
    \item \textbf{YOLOv11s:} Mặc dù chỉ nhỉnh hơn YOLOv8s 0.001 về mAP@50, nhưng có hiệu suất vượt trội ở lớp Scissors (0.983) - lớp dễ nhất. Đồng thời duy trì Rock ở mức cao (0.956).

    \item \textbf{YOLOv8s:} Cân bằng tốt giữa các lớp, không có lớp nào quá nổi trội hay quá yếu. Scissors (0.977) và Rock (0.948) cho kết quả tốt.

    \item \textbf{YOLOv9s:} Có mAP@50 thấp nhất (0.950), nhưng lại đạt kết quả tốt nhất cho Paper (0.942) và Rock (0.958). Điều này cho thấy YOLOv9 có thể phù hợp hơn với các lớp có đặc trưng ít nổi bật.
\end{itemize}

\textbf{Kết luận:}

Đường cong PR xuất sắc của cả ba mô hình (AUC > 0.95) xác nhận rằng kiến trúc YOLO rất phù hợp với bài toán nhận diện cử chỉ tay. Sự khác biệt nhỏ giữa các mô hình (<0.5\%) cho thấy cả ba thế hệ đều đã đạt mức độ tối ưu cao. YOLOv11s với mAP@50 cao nhất (0.955) kết hợp với F1-Score tốt nhất (0.936) từ phân tích trước đó, khẳng định vị trí dẫn đầu về hiệu suất tổng thể.

\subsubsection{Kết quả Dự đoán Trên Test Set}

Hình \ref{fig:predictions} minh họa một số kết quả dự đoán của YOLOv11s trên tập test:

\begin{figure}[H]
    \centering
    \begin{subfigure}[b]{0.48\textwidth}
        \centering
        \includegraphics[width=\textwidth]{../evaluation_results/YOLOv11/val_batch0_pred.jpg}
        \caption{Batch 0}
    \end{subfigure}
    \hfill
    \begin{subfigure}[b]{0.48\textwidth}
        \centering
        \includegraphics[width=\textwidth]{../evaluation_results/YOLOv11/val_batch1_pred.jpg}
        \caption{Batch 1}
    \end{subfigure}
    \caption{Ví dụ kết quả dự đoán của YOLOv11s trên test set}
    \label{fig:predictions}
\end{figure}


Hình \ref{fig:predictions} minh họa kết quả dự đoán của YOLOv11s trên hai batch ngẫu nhiên từ tập test, cho thấy hiệu suất thực tế của mô hình trong các tình huống đa dạng:

\textbf{Độ tin cậy cao:} Hầu hết các detection đạt confidence score từ 0.8-0.9, phản ánh độ chắc chắn cao của mô hình khi đưa ra dự đoán. Điều này phù hợp với chỉ số Precision >94\% đã đo được.

\textbf{Xử lý đa dạng bối cảnh:} Mô hình hoạt động tốt trên nhiều điều kiện thực tế khác nhau:
\begin{itemize}
    \item Môi trường đa dạng: phòng khách, bếp, ngoài trời, hồ bơi
    \item Điều kiện ánh sáng khác nhau: tự nhiên, nhân tạo, backlight
    \item Góc chụp và khoảng cách khác nhau
    \item Nhiều đối tượng (người) trong cùng một khung hình
\end{itemize}

\textbf{Ứng dụng thực tế:} Các hình ảnh test bao gồm screenshot từ video call, YouTube, và các nguồn video thực tế, chứng minh khả năng triển khai của mô hình trong các ứng dụng tương tác người-máy như điều khiển bằng cử chỉ, game, hoặc giao tiếp không tiếp xúc.

\textbf{Giới hạn quan sát được:} Một số trường hợp có confidence thấp hơn (0.8) thường xuất hiện khi có motion blur, tay bị che khuất một phần, hoặc góc chụp nghiêng. Tuy nhiên, mô hình vẫn đưa ra dự đoán đúng trong hầu hết các trường hợp này.
